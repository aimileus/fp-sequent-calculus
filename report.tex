\documentclass[12pt,a4paper]{article}
\input{latexmacros.tex}

\title{Sequent calculus proof generation in Haskell}
\author{Jonas van der Schaaf, Emiel Wiedijk}
\date{\today}
\hypersetup{pdfauthor={Jonas van der Schaaf, Emiel Wiedijk}, pdftitle={Sequent Calculus proof generation in Haskell}}

\begin{document}

\maketitle

\begin{abstract}
    Sequent calculus is a way to formalize natural deduction proofs. We give a
    Haskell library for the generation of sequent Calculus sroofs. Our code can
    generate proofs for both classical and intuitionistic sequent calculus. We
    believe that the extensive use of typeclasses allows our code to be used to
    generate proofs for many other sequent calculi as well. We also support
    exporting of a sequent calculus proof to \LaTeX{} code using the package
    ebproof.
\end{abstract}

\newpage
\tableofcontents

\clearpage

% We include one file for each section. The ones containing code should
% be called something.lhs and also mentioned in the .cabal file.

\input{lib/Sequent.lhs}

\input{lib/PropSeq.lhs}

\input{lib/InSeq.lhs}

\input{lib/ModalSeq.lhs}

\input{lib/ZipperSequent.lhs}

\input{lib/Latex.lhs}

\newpage
\input{test/Main.lhs}

\newpage
\input{lib/Utils.lhs}
\addcontentsline{toc}{section}{Bibliography}
\bibliographystyle{alpha}
\bibliography{references.bib}
\end{document}
